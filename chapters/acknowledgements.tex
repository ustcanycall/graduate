\begin{acknowledgements}
不知不觉博士生涯就要结束了,在这五年半的时间里,我在许多城市,包括合肥,苏州,北京,上海等进行学习,我遇到了很多帮助我的老师和同学,在这里,我向他们表达最真诚的谢意。

首先要感谢我的导师周学海教授,前两年半的博士生涯,是周老师一步一步带着我进行体系结构的学习,带我进入了计算机体系结构的大门,让我领略了体系结构的魅力。周老师让不管在生活上还是在研究上都充满了大智慧,特别是在研究上,能够在high-level的角度一针见血地点出问题的本质。周老师在2016年年初将我推荐到计算所陈老师的国重实验室(智能处理器研究中心)进行深入学习,使我能够接触最前沿的科学研究,非常感谢周老师当时的决定。

其次,我要感谢计算所的陈云霁研究员。陈老师学识渊博,平易近人,能够处处为学生着想,为学生创造一个非常良好的学习和工程的环境。我还记得当时和陈老师一起通宵写课题,一起爬山,打桌游,吃火锅的时光。在北京计算所的一年时间里,我学到了许多知识,包括工程方面,如前端设计,前端验证,驱动编写等;还包括科研方面,包括神经网络算法知识,性能分析,加速器架构设计等。

同时我要感谢杜子东老师。在2017年,我加入了杜老师的前瞻组,并在接下来的一年半时间里在上海寒武纪分部专心进行科研。杜老师领我进入了科研的大门,帮我培养学术上的思维习惯,教我如何选择idea,组织文章结构,呈现硬件架构,呈现实验结果,进行性能分析,能耗分析等,从而完成一篇高质量的论文。杜老师为人师表,在科研上孜孜不倦,认真负责,在杜老师的帮助下,最终让我达到了毕业的要求。

另外,我要感谢嵌入式系统实验室的李曦老师,陈香兰老师,王超老师对我的指导。同时还要感谢寒武纪陈天石老师,刘少礼老师,王在老师,喻歆老师,郭琦老师,刘道福老师,罗韬老师,支天老师和软件所的李玲老师等对我生活上,工程上和学术上的帮助。

同时,感谢嵌入式系统实验室的师兄师弟们,特别感谢周金红师兄,余奇,马翔,万波,李俊,陈航,赵勇对我的帮助。感谢在寒武纪和我一起工作的同学们:兰慧盈,周聖元,刘雨辰,韩栋,李震,张士锦等对我的照顾,跟大家一起学习,做工程,做研究真的学到了很多。

最后,感谢我的父母和亲人,成长就是最好的报答。

\end{acknowledgements}
